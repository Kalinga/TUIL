\documentclass[12pt]{article}
\usepackage[T1]{fontenc}
\usepackage{hyperref}   % hypertext links, URLs
\usepackage{algorithm}
\usepackage{algorithmic}
\usepackage{multicol}
\usepackage[utf8]{inputenc}

\usepackage{amsmath}
\usepackage{amsfonts}


\setlength{\baselineskip}{16.0pt}    % 16 pt usual spacing between lines
\setlength{\parskip}{3pt plus 2pt}
\setlength{\parindent}{10pt}
\setlength{\oddsidemargin}{0.5cm}
\setlength{\evensidemargin}{0.5cm}
\setlength{\marginparsep}{0.75cm}
\setlength{\marginparwidth}{2.5cm}
\setlength{\marginparpush}{1.0cm}
\setlength{\textwidth}{150mm}

\newcommand{\N}{{\mathbb N}}
\newcommand{\Z}{{\mathbb Z}}
\newcommand{\Q}{{\mathbb Q}}
\newcommand{\R}{{\mathbb R}}
\renewcommand{\Pr}{{\mathbf{Pr}}}
\newcommand{\E}{{\mathbf{E}}}

\renewcommand{\gcd}{\text{gcd}}

\newcommand{\ceil}[1]{\lceil#1\rceil}

\newcommand{\floor}[1]{\lfloor#1\rfloor}

\begin{document}
%\begin{multicols}{1}
\begin{center}
{\large Variation of Euclidean Algorithm for finding gcd of two numbers} \\
\copyright 2017 by Kalinga Bhusan Ray \\
November 15, 2017 \\
Technische Universität Ilmenau, Ilmenau, Germany \\
\author{Kalinga Bhusan Ray}
\end{center}

\section*{Acknowledgement}
Dietzfelbinger, Martin (Professor, TU Ilemanu) for proofreading and valuable feedback
%\end{multicols}

\quad

\begin{multicols}{2}
\section{Abstract}
Finding greatest common divisor(gcd) is very basic to the elementary mathematics.
Mathematics of calculating gcd and it properties has been studied over the several thousand years. And it still fascinates today's mathematicians and computer scientisct 
because of its application in modern cryptography which is essential to world wide web
secure communication and other secure data transmission.
I don't think the findings proposed in this paper is not noted in the past, however i have not seen it anywhere and not in the text books discussing Euclidean Algorithm for finding gcd. And this motivated me to document this finding. 

\section{Keywords}
GCD Algorithm, greatest common divisor, Euclidean Algorithm for gcd, gcd

% [t] indicates the top of the page
% {table*} used for wide table
\begin{table*}[t]
\centering
	\begin{tabular}{c c c c c c}
		... & -5 & 0 & 5 & 10 & ... \\
		... & -4 & 1 & 6 & 11 & ... \\
		... & -3 & 2 & 7 & 12 & ... \\
		... & -2 & 3 & 8 & 13 & ... \\
		... & -1 & 4 & 9 & 14 & ...
	\end{tabular}
\caption{Table showing equivalence class for modulo 5 on integers}
\label{table:1}
\end{table*}

\section{Introduction}
Greatest common divisor(gcd) of two integers $a$ and $b$ can be defined as: A common divisor of $a$ and $b$ is a greatest integer that divides both. The gcd is the unique	nonnegative $c$ such that $c$ is a common divisor, and if $d$ is a common divisor then $d$ divides $c$; ensuring $c$ is the highest among all such $d$. 
Gcd for two numbers exhibits some interesting behavior and its worth noting them here.
$\gcd(a,b)=\gcd(b,a)$ and $\gcd(-a,b)=\gcd(a,b)$ and $\gcd(a,0)=a$
and $\gcd(a,b)=\gcd(a-ub,b)$, for arbitrary $u\in\Z$

The methodology for finding that, was initially formulated by the famous Greek mathematician Euclid. This methodology for finding greatest common divisor for two natural numbers is otherwise known as Euclidean Algorithm for finding gcd of two 
natural numbers. The definition for the Euclidean Algorithm can be seen as below.
\newline \newline \textbf {Euclid's Rule} \textit {If x and y are positive integers with x $\geq$ y,
 then} \break \vspace{0mm} \hspace{4cm} gcd(x, y) = gcd(x mod y, y).
\newline 
\newline An algorithm based on above definition can be found from text book [1] as follow.
\newline \textbf{Algorithm 1:}
\newline \textbf {Euclid's algorithm for finding the greatest common divisor of two numbers.}\newline \underline {function Euclid}(a, b)
\newline Input: Two integers a and b with a $\geq$ b $\geq$ 0 
\newline Output: gcd(a, b)
\newline
\newline if b=0: return a
\newline return Euclid(b, a mod b)
\newline 
\newline From the above compact algorithm, it can be noticed, the algorithm is based on recursive approach to solve the problem and usage of mathematical modulus operator(\textit {mod} in short).
For two integers \texttt{a} and \texttt{b}, the working of modulus operator can be explained as follows.
For \texttt{a} > \texttt{b}, \texttt{a} mod \texttt{b} = \texttt{r},  such that \texttt{b}>\texttt{r} and  \texttt{a} = \texttt{q} * \texttt{b} + \texttt{r}, \texttt{q} is the quotient of the division of \texttt{a} and \texttt{b} and the \texttt{r} is the remainder of the said division. When the remainder is \texttt{0}, then \texttt{a} is said to be divisible by \texttt{b}.

So from the mentioned algorithm it is quite evident that, the remainder (\texttt{r}) of the modulus operation is being treated as one of the input of subsequent recursive function call, and also happens to be the \textbf{smallest} of the two parameters.
The main purpose of this paper is to show whether this number \texttt{r} is unique for our gcd algorithm purpose, or there exists another number that can also be substituted for this number?

\section{Methodology}
% [t] indicates the top of the page
\begin{table*}[t]
\centering
	\begin{tabular}{c | c | c}
		\hline
		$i$ & $a\textsubscript{i}$ & $b\textsubscript{i}$ \\ [0.5ex] 
		\hline
		0 & 12742 & 10534 \\ 
		1 & 10534 & 2208 \\  
		2 & 2208  & 1702 \\
		3 & 1702  & 506 \\
		4 & 506   & 184 \\
		5 & 184   & 138 \\
		6 & 138   & 46 \\
		7 & 46    & 0
	\end{tabular}
\caption{Table showing intermediate values while calculating gcd for (10534, 12742)}
\label{table:2}
\end{table*}
In the previous section, we have seen how modular arithimatic is crucial to the computation of $\gcd$. There is a very interesting property exhibited by modular arithmetic.
When we perform modular arithmetic such as \texttt{a} mod \texttt{b}, there exists
\texttt{b} number of equivalence classes, and all elements in one equivalence give same result on modulo operation with \texttt{b} and such equivalent element can be substituted for each other when performing modulo \texttt{b}.
\newline
Considering an actual value for \texttt{b} to be 5, we can have below 5 equivalence classes.

And from the table number:1, it is clear that for modulo 5, the numbers 7 and 12 behaves exactly same. And mathematically they as denoted as congruent to each other for modulo 5.
Symbolically it can be represented as, \texttt{7} $\equiv$ \texttt{12} (mod \texttt{5}).
So when, the modulo \texttt{b} give same \texttt{r} for a set of different numbers, finding and studying the equivalence for \texttt{r} was enticing.

Yes, a closer look at the way modulus operator works gives the answer, there exist another  number \texttt{s}, \texttt{s} $\neq$ \texttt{r}, that can be substituted in the algorithm involving further recursive modulus, without affecting the final result. And the number \textit{s} can be expressed as below
\newline $s = -(a-\ceil{a/b}*b) = \ceil{a/b}*b - a$
\newline \textit{r} can be expressed similarly as below
\newline $r = a - \floor{a/b}*b$
\newline This paper intents to show and proof that \textit{r}, \textit{s} $\in \N$, \textit{r}, \textit{s} $<$ \textit{b}, and can replace each other when used as one of the input to the algorithm susequently. Mathematically, $\gcd(a,b) =\gcd(b,r) = \gcd(b,s)$ to be proven.

Let's look at a more detailed algorithm, iterative in nature and presented in a book by M. Dietzfelbinger, and is based on the previously mentioned Euclid's algorithm which is recursive in nature but the principle remains same.
\newline
\newline \textbf{Algorithm 2:}
\newline \textbf{GCD Algorithm (Euclidean Algorithm)}
\newline INPUT: Two integers n, m
\newline METHOD:

%\algsetup{
%	indent=1em,	
%	linenosize=\small,
%	linenodelimiter=.
%}

\begin{algorithmic}[1]
\STATE a, b: integer
\IF{$|n| \geq |m|$}
	\STATE $a \gets |n|; b \gets |m|;$
\ELSE
	\STATE $b \gets |m|; a \gets |n|;$
\ENDIF
\WHILE{$b > 0$ } 
	\STATE $(a, b) \leftarrow (b, a  $ $mod $ $b);$
\ENDWHILE
\RETURN a
\end {algorithmic}

\begin{table*}[t]
\centering
	\begin{tabular}{c | c | c | c}
		\hline
		$i$ & $a\textsubscript{i}$ & $b\textsubscript{i} $ $using $ $s$ & $corresponding $ $r$\\ [0.5ex] 
		\hline
		0 & 12742 & 10534 & na  \\ 
		1 & 10534 & 8326  & 2208\\  
		2 & 8326  & 6118  & 2208\\
		3 & 6118  & 3910  & 2208\\
		4 & 3910   & 1702 & 2208\\
		5 & 1702   & 1196 & 506\\
		6 & 1196   & 690  & 506\\
		7 & 690    & 184  & 506\\
		8 & 184    & 46   & 138\\
		9 & 46     & 0    & na
	\end{tabular}
\caption{Table showing intermediate values while calculating gcd for (10534, 12742)}
\label{table:3}
\end{table*}

Let's consider the initial values for the variables \texttt{a} and \texttt{b} populated in the line 3 and 5 from the input values to the algorithm $n$, and $m$ be (a\textsubscript{0}, b\textsubscript{0}). In the line number 8 it can be noticed that; for each iteration in the loop we have new sequence of (a, b) and for the i\textsuperscript{th} iteration we can formulate a\textsubscript{i} = b\textsubscript{i-1}, b\textsubscript{i} = a\textsubscript{i-1} mod b\textsubscript{i-1}, for $1 \geq i \geq t$, and b\textsubscript{t} = 0

For \textbf{t} = 7, b\textsubscript{t} = 0 and the algorithm exits with the return value 
46 as the gcd for (12742, 10534).
% ---------------------------------TABBLE 4 -----------------------------------
\begin{table*}[t]
\centering
	\begin{tabular}{c | c | c | c| c}
	\hline
	$i$ & $a\textsubscript{i}$ & $b\textsubscript{i} = min(r,s)$ & $r$ & $s$\\ [0.5ex] 
	\hline
		0 & 12742 & 10534 & na   & na \\ 
		1 & 10534 & 2208  & 2208 & 8326\\  
		2 & 2208  & 506   & 1702 & 506\\
		3 & 506   & 184   & 184  & 322\\
		4 & 184   & 46    & 138  & 46\\
		5 & 46    & 0     & na   & na
	\end{tabular}
\caption{Table showing intermediate values while calculating gcd for (10534, 12742)}
\label{table:4}
\end{table*}
% ---------------------------------TABBLE 4 -----------------------------------

Now it is the time to see the results when the claim made in this paper is being applied 
to the algorithm. According to the claim, instead of the remainder of the division, the seek value (\texttt{s}) that could make the dividend a multiple of divisor, which is same as difference between divisor and remainder, can be used in the algorithm without affecting the result. So we can modify line no. 8 of the Algorithm 2 by using a modified implementation of \texttt{mod} and we call it \texttt{mod'} and can be expressed as below. Let's name this modified algorithm as algorithm no: 3 .  \linebreak
\textit{a} mod' \textit{b} = \textit{s}, where  \textit{s} = \textit{b - r}, \textit{r} = \textit{a} mod \textit{b}

The values obtained during the intermediate steps captured for the algorithm no: 3 in table:3 is presented for the reference.

If we compare the contents of table 2 and 3, we could see some common numbers and some of them differs. One could also observe that in the table 3 we have two more entries, that's because the algorithm takes two more recursion to complete. This is because for 
this combination of numbers we have received 6 times \texttt{s} > \texttt{r} and 2 times 
\texttt{s} < \texttt{r}. But for other combination of numbers, the scenario could be different.

\textbf{Alternative Approach using 's'}
Now the question raises, what is the significance of this \texttt{s} over \texttt{r}.
If we analyse the Euclidean algorithm for finding greatest common divisor, it is quite evident that, the arguments are getting reduced and one of them reduced till 0 when finally the output of the algorithm is obtained. So it is quite natural that if \texttt{s} can be substituted  for \texttt{r} then finding the smallest of them at each recursive call and using it for the subsequent recursion can help the recursion bringing the operand closer to 0 more quickly. The modified algorithm can be presented below.
\newline \textbf{Algorithm 4:}
\begin{algorithmic}[1]
\STATE a, b: integer
\IF{$|n| \geq |m|$}
	\STATE $a \gets |n|; b \gets |m|;$
\ELSE
	\STATE $b \gets |m|; a \gets |n|;$
\ENDIF
\WHILE{$b > 0$ } 
	\STATE $(a, b) \leftarrow (b, a  $ $mod'' $ $b);$
\ENDWHILE
\RETURN a
\end {algorithmic}

So it can be noted from the above that there is apparent no change from the previously given algorithm except line no. 8. The modified algorithm uses $mod'$ instead of \texttt{mod} for the modulus operation. This $mod''$ can be expressed as below.
If \textit{r} = \textit{a} mod \textit{b}, \textit{s} = \textit{b - r};
\textit{a} $mod''$ \textit{b} = min(r, s), where \textit{min} is the function returns the least of the arguments.

The values obtained during the intermediate steps captured for the algorithm no: 3 in following table is presented for the reference.

\section{Conclusion}
From the table 4 contents it is clear that by using the \texttt{min(r,s)} is advantageous and bring down the number of recursion. And also it is proved that we have a substitute for \texttt{r} in the context of modulus arithmetic and which can be used within the intermediate steps of the algorithm to find gcd of two numbers. The order of optimization is still under investigation and could be significant when we integers with very very long digits may be in the order of thousands! Having discussed the advantage of \texttt{$mod''$}, it also should be noted that \texttt{$mod''$} is implemented with the help of existing \texttt{mod} operator and an additional subtraction. The running time of subtraction is linear however the running time of modular operation is quadratic in nature. So this additional subtraction can be well compensated by reduction in the number of intermediate steps in the recursion. There are also other highly optimised algorithm exists such as \texttt{Binary GCD}, \texttt{Lehmer's Algorithm} etc. for finding the gcd and this paper does not intent to provide a optimise algorithm for gcd.

\begin{thebibliography}{6}

\bibitem{latex}S. Dasgupta, C. Papadimitriou and U. Vazirani, \textsl{Algorithms
},
first edition, Mc Graw Hill(2008).
\bibitem{latex}Martin Dietzfelbinger \textsl{Primality Testing in Polynomial Time
},
first edition, Springer(2004).

\bibitem{website}The GNU Multiple Precision Arithmetic Library \url{https://gmplib.org/manual/Greatest-Common-Divisor-Algorithms.html (Accessed 15th Nov' 2017)}.

\end{thebibliography}

\end{multicols}
\end{document}