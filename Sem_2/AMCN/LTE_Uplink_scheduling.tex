%\documentclass[12pt]{article}
\documentclass[conference]{IEEEtran}
\IEEEoverridecommandlockouts

\usepackage[T1]{fontenc}
\usepackage[utf8]{inputenc}
\usepackage{hyperref}   % hypertext links, URLs

% *** SPECIALIZED LIST PACKAGES ***
%
%\usepackage{algorithmic}
% algorithmic.sty was written by Peter Williams and Rogerio Brito.
% This package provides an algorithmic environment fo describing algorithms.
% You can use the algorithmic environment in-text or within a figure
% environment to provide for a floating algorithm. Do NOT use the algorithm
% floating environment provided by algorithm.sty (by the same authors) or
% algorithm2e.sty (by Christophe Fiorio) as the IEEE does not use dedicated
% algorithm float types and packages that provide these will not provide
% correct IEEE style captions. The latest version and documentation of
% algorithmic.sty can be obtained at:
% http://www.ctan.org/pkg/algorithms
% Also of interest may be the (relatively newer and more customizable)
% algorithmicx.sty package by Szasz Janos:
% http://www.ctan.org/pkg/algorithmicx

%\usepackage{algorithm}
\usepackage{algorithmic}

%\usepackage{multicol}


% *** MATH PACKAGES ***
%
%\usepackage{amsmath}
% A popular package from the American Mathematical Society that provides
% many useful and powerful commands for dealing with mathematics.
%
% Note that the amsmath package sets \interdisplaylinepenalty to 10000
% thus preventing page breaks from occurring within multiline equations. Use:
%\interdisplaylinepenalty=2500
% after loading amsmath to restore such page breaks as IEEEtran.cls normally
% does. The latest version and documentation can be obtained at:
% http://www.ctan.org/pkg/amsmath
\usepackage{amsmath}
\usepackage{amsfonts}
\interdisplaylinepenalty=2500

% *** ALIGNMENT PACKAGES ***
%
%\usepackage{array}
% Frank Mittelbach's and David Carlisle's array.sty patches and improves
% the standard LaTeX2e array and tabular environments to provide better
% appearance and additional user controls. As the default LaTeX2e table
% generation code is lacking to the point of almost being broken with
% respect to the quality of the end results, all users are strongly
% advised to use an enhanced (at the very least that provided by array.sty)
% set of table tools. array.sty is already installed on most systems. The
% latest version and documentation can be obtained at:
% http://www.ctan.org/pkg/array


% IEEEtran contains the IEEEeqnarray family of commands that can be used to
% generate multiline equations as well as matrices, tables, etc., of high
% quality.
\usepackage{array}

% *** CITATION PACKAGES ***
%
\ifCLASSOPTIONcompsoc
  % IEEE Computer Society needs nocompress option
  % requires cite.sty v4.0 or later (November 2003)
  \usepackage[nocompress]{cite}
\else
  % normal IEEE
  \usepackage{cite}
\fi

\hyphenation{op-tical net-works semi-conduc-tor}

\setlength{\baselineskip}{16.0pt}    % 16 pt usual spacing between lines
\setlength{\parskip}{3pt plus 2pt}
\setlength{\parindent}{10pt}
\setlength{\oddsidemargin}{0.5cm}
\setlength{\evensidemargin}{0.5cm}
\setlength{\marginparsep}{0.75cm}
\setlength{\marginparwidth}{2.5cm}
\setlength{\marginparpush}{1.0cm}
\setlength{\textwidth}{150mm}

\newcommand{\N}{{\mathbb N}}
\newcommand{\Z}{{\mathbb Z}}
\newcommand{\Q}{{\mathbb Q}}
\newcommand{\R}{{\mathbb R}}
\renewcommand{\Pr}{{\mathbf{Pr}}}
\newcommand{\E}{{\mathbf{E}}}

\newcommand{\var}[1]{\textit{#1}}

\newcommand{\progvar}[1]{\texttt{#1}}

\renewcommand{\mod}{\text{mod}}

\newcommand{\moddash}{\text{mod'}}

\newcommand{\moddashdash}{\text{mod''}}

\renewcommand{\gcd}{\text{gcd}}

\newcommand{\ceil}[1]{\lceil#1\rceil}

\newcommand{\floor}[1]{\lfloor#1\rfloor}

%--------------------------------------------ENV Setup-----------------------------

\begin{document}
%\begin{multicols}{1}
%
% paper title
% Titles are generally capitalized except for words such as a, an, and, as,
% at, but, by, for, in, nor, of, on, or, the, to and up, which are usually
% not capitalized unless they are the first or last word of the title.
% Linebreaks \\ can be used within to get better formatting as desired.
% Do not put math or special symbols in the title.
\title {Survey on Uplink Scheduling Techniques in LTE and LTE-A
Networks: M2M, D2D and V2V Perspective} 

% author names and affiliations
% use a multiple column layout for up to three different
% affiliations
\author{
%\IEEEauthorblockN{Kalinga Bhusan Ray}
%\IEEEauthorblockN {Kalinga Bhusan Ray}
%\IEEEauthorblockA{Research in Computer System and Engineering\\
%Technische Universität Ilmenau\\
%Ilmenau, Germany\\
%Email: kalinga-bhusan.ray@tu-ilmenau.de}
\begin{tabular}[t]{c@{\extracolsep{2em}}c@{\extracolsep{2em}} c@{\extracolsep{2em}}c}
Iresh Dudeja  & Kalinga  Bhusan Ray & Mohamud & qurrat-ul Ainy \\
iresh.dudeja@  & kalinga-bhusan.ray@ & muhammad-attahir.jibril@ &  qurrat-ul.ainy@ \\ 
tu-ilmenau.de & tu-ilmenau.de & tu-ilmenau.de & tu-ilmenau.de \\
\end{tabular}
}

%\end{multicols}

% make the title area
\maketitle

\begin{abstract}
With the advancement of technology and appropriateness of OFDM, it is used as  prefered technology for downlink communication between User Equipment and evolved NodeB (eNodeB). However the same scheme is not helpful in the uplink scenario because of its inherent property of high peak to average power ratio (PAPR). High PAPR is not suitable for battery powered low wattage equipment such as cellular phones. This restriction gave rise the possibility of using the single carrier frequency division multiple access (SC-FDMA) which has low PAPR attribute. Selection of SC-FDMA comes with a difficult constraint to be maintained all the time, resources allocated for a particular user must be continuous in nature. Keeping this constraints, many algorithms were proposed to maximize the system throughput at the same time  being fair to all users located differently. Mainly these algorithms are based on channel dependent and proportional fairness paradigms. 
\end{abstract}

% no keywords
%GCD Algorithm, greatest common divisor, Euclidean Algorithm for gcd, gcd

\section{Introduction}
KALINGA
\section{section2}
 qurrat-ul.ainy
\section{section3}
IRESH
\section{section4}
Muhammad Attahir Jibril
\section{Conclusion}

% use section* for acknowledgment
\ifCLASSOPTIONcompsoc
  % The Computer Society usually uses the plural form
  \section*{Acknowledgments}
\else
  % regular IEEE prefers the singular form
  \section*{Acknowledgment}
\fi
The author would like to thank xxx, Department of Computer Science and Automation of Technische Universität Ilmenau for proofreading and providing his valuable feedback.

\begin{thebibliography}{6}

\bibitem{latex}Author names, \textsl{Book name
},
first edition, Mc Graw Hill(2008).
\bibitem{latex}Author1 \textsl{Book name
},
first edition, Springer(2004).

\bibitem{website}The GNU Multiple Precision Arithmetic Library \url{https://gmplib.org/manual/Greatest-Common-Divisor-Algorithms.html (Accessed 15th Nov' 2017)}.

\end{thebibliography}

%\end{multicols}
\end{document}